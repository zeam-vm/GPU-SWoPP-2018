\documentclass[submit,techrep]{ipsj}



\usepackage{booktabs} % For formal tables
\usepackage{xcolor}

\usepackage[dvips]{graphicx}
\usepackage{latexsym}

\def\Underline{\setbox0\hbox\bgroup\let\\\endUnderline}
\def\endUnderline{\vphantom{y}\egroup\smash{\underline{\box0}}\\}
\def\|{\verb|}

\setcounter{巻数}{53}%vol53=2012
\setcounter{号数}{10}
\setcounter{page}{1}


\def\tightlist{\itemsep1pt\parskip0pt\parsep0pt}

\begin{document}
\title{Elixirプログラミングにおける超並列化を実現するためのGPGPU活用手法}
\etitle{A Method Using GPGPU for Super-Parallelization in Elixir Programming}

\paffiliate{Kitakyu-u}{北九州市立大学\\
University of Kitakyushu}
\affiliate{Delight}{有限会社デライトシステムズ\\
Delight Systems Co., Ltd.}
\paffiliate{Kyoto-u}{京都大学\\
Kyoto University}


\author{山崎 進}{Yamazaki Susumu}{Kitakyu-u}[zacky@kitakyu-u.ac.jp]
\author{森 正和}{Mori Masakazu}{Delight}[mori@delightsystems.com]
\author{上野 嘉大}{Ueno Yoshihiro}{Delight}[delightadmin@delightsystems.com]
\author{高瀬 英希}{TAKASE Hideki}{Kyoto-u}[takase@i.kyoto-u.ac.jp]

\begin{abstract}
ElixirではFlowというMapReduceに近いモデルの並列プログラミングライブラリが広く普及している。Flowを用いるとパイプライン演算子を用いた簡潔な表現でマルチコアCPUの並列性を活用することができる。我々はFlowによるプログラム記述がGPGPUにも容易に適用できるという着想を得て,OpenCLによるプロトタイプを実装した。現行のGPUの一般的なアーキテクチャであるSIMDでは,単純な構造で均質で大量にあるデータを同じような命令列で処理する場合に効果を発揮する。一方,Flowでは,単純な構造で均質で大量にあるデータであるリスト構造に対し,パイプライン演算子でつながれた一連の命令列で処理する。そこで,この一連の命令列をひとまとめにしてGPU向けにコンパイルし,入力となるリスト構造を配列データにまとめてコードとともにGPUに転送して実行することで高速化を図れるというのが我々の着想である。そこで,素体のロジスティック写像を用いたベンチマークプログラムを開発し,期待される性能向上がどのくらいになるかを評価した。Mac Pro (Mid 2010)とATI Radeon HD 5770 で構成されたシステムで評価した。言語は Elixir とC言語で比較し,逐次実行と並列実行,CPU/GPU利用,インライン展開の有無などで比較した。その結果,Elixirで実行した場合と比べてGPGPUを利用すると10倍以上の速度向上が期待できることがわかった。今後,LLVMを用いてコード生成器を含む処理系を開発する予定である。
\end{abstract}


\begin{jkeyword}
Elixir,C++,Node.js,マルチタスク
\end{jkeyword}


\begin{eabstract}
Elixir has Flow that is a popular parallel programming library similar to MapReduce. Flow can realize parallel programming on multi-core CPUs by simple description using pipeline operators. We ideate that code description of Flow can be applied to GPGPU easily, and implements prototypes. The SIMD architecture that current general GPUs adopt is effective when a single instruction sequence processes a simple, homogeneous and mass data structure, and code using Flow is a single instruction sequence connected by pipeline operators that processes a linked-list structure, which is asimple, homogeneous and mass data structure. Thus, we propose a code optimization method by compiling the instruction sequence for the GPU, sending it and a mass array composed of data from the linked-list, and executing them. We develop a benchmark suit of the logistic maps over prime fields, and evaluate performance of the proposal system using it. We evaluate it using a system including Mac Pro (Mid 2010) and ATI Radeon HD 5770. We compare Elixir and the C language, sequential and parallel execution, CPU and GPU, and with/without inlining. The results shows that the benchmark using C languages and GPGPU is over 10 times faster than that using Elixir and CPU. We plan to develop a processing system including a code generator using LLVM.
\end{eabstract}

\begin{ekeyword}
Elixir, C++, Node.js, multi-tasking
\end{ekeyword}

\maketitle

\input{description}

\begin{acknowledgment}

\end{acknowledgment}


\bibliographystyle{ipsjsort}
\bibliography{reference}

\end{document}
